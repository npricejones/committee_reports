
% Default to the notebook output style

    


% Inherit from the specified cell style.




    
\documentclass[11pt]{article}

    \usepackage{natbib}
    \usepackage[usenames, dvipsnames]{color}
    %\bibliographystyle{apj}
    
    \usepackage[T1]{fontenc}
    % Nicer default font (+ math font) than Computer Modern for most use cases
    \usepackage{mathpazo}

    % Basic figure setup, for now with no caption control since it's done
    % automatically by Pandoc (which extracts ![](path) syntax from Markdown).
    \usepackage{graphicx}

    \usepackage{geometry} % Used to adjust the document margins
    \usepackage{amsmath} % Equations
    \usepackage{amssymb} % Equations
    
    \usepackage[mathletters]{ucs} % Extended unicode (utf-8) support
    \usepackage[utf8x]{inputenc} % Allow utf-8 characters in the tex document
    \usepackage{fancyvrb} % verbatim replacement that allows latex
    \usepackage{grffile} % extends the file name processing of package graphics 
                         % to support a larger range 
    % The hyperref package gives us a pdf with properly built
    % internal navigation ('pdf bookmarks' for the table of contents,
    % internal cross-reference links, web links for URLs, etc.)
    \usepackage{hyperref}
    \usepackage[normalem]{ulem} % ulem is needed to support strikethroughs (\sout)
                                % normalem makes italics be italics, not underlines
    
    \usepackage{fancyhdr}
    \usepackage{subcaption}  
    \usepackage{wrapfig,booktabs}
 
	\setcounter{topnumber}{2}
	\setcounter{bottomnumber}{2}
	\setcounter{totalnumber}{4}
	\renewcommand{\topfraction}{0.85}
	\renewcommand{\bottomfraction}{0.85}
	\renewcommand{\textfraction}{0.15}
	\renewcommand{\floatpagefraction}{0.8}
	\renewcommand{\textfraction}{0.1}
	\setlength{\floatsep}{5pt plus 2pt minus 2pt}
	\setlength{\textfloatsep}{5pt plus 2pt minus 2pt}
	\setlength{\intextsep}{2.5pt plus 0pt minus 0pt}
    
    \geometry{verbose,tmargin=1in,bmargin=1in,lmargin=1in,rmargin=1in}
    
    
    \pagestyle{fancy}
    \fancyhf{}
    \rhead{Natalie Price-Jones}
    \lhead{March 2019 Committee Report}
    \rfoot{Page \thepage}


    \begin{document}
    
    \section*{Committee Report}
    
    The primary goal of my PhD work is to answer questions about the Milky Way's formation and evolution using chemical tagging. Since my last meeting, I have completed and submitted a paper demonstrating that it's possible to recover realistic simulated clusters with a density-based approach to clustering in chemical space.    

    \subsection*{Degree progress}

	Over the past six months, I have been focussed on applying DBSCAN (density based spatial clustering applications with noise - \citealt{Ester1996}) to simulated chemical spaces and making tweaks to my simulation and the algorithm to get the best results. Most of my time has been spent working on a paper on this subject, and I have nearly completed the submission draft. The simulated chemical spaces in the paper emulate results from H-band spectroscopic survey APOGEE (Apache Point Observatory Galactic Evolution Experiment - \citealt{Majewski2017}), but the methods are widely applicable. As initially planned, I separately use spectra and abundances as chemical spaces to search for clusters, and make modifications to both to improve cluster finding performance. In the case of abundances, I test cluster recovery for several different choices of measurement noise, ranging from the current abundance uncertainties quoted by APOGEE in \citet{Abolfathi2018} to the theoretical minimum uncertainties for abundances derived from APOGEE spectra \citep{Ting2016b}. In the case of spectra, I use principal component analysis to reduce the dimensionality of the high-resolution H-band spectra from 7214 to 30 - this has the effect of reducing the importance of continuum pixels without value for chemically distinguishing stars. Taking these steps has allowed me to investigate the performance of DBSCAN in a variety of chemical spaces.
	
	In my initial thesis proposal, I had suggested two simple statistics to measure the success of cluster recovery; the homogeneity and the completeness of the recovered clusters. Homogeneity measures the extent to which an identified group is dominated by a particular input cluster. Completeness measures the amount of input cluster members that ended up in a single identified group. Both statistics are important, because DBSCAN identified groups should be dominated by a single input cluster (high homogeneity), and not split up that input cluster (high completeness). However a measurement of these quantities assumes that the true cluster membership is already known. While these statistics might measure the success of a clustering algorithm on my simulated data, they would not be relevant when clustering on real data. One way around this problem is to make use of open cluster data, since these groups are similar to the clusters I generate, but their membership is known even in observed data. To quantify the success of DBSCAN, I measured the recovery fraction of the open clusters from the Open Cluster Chemical Abundances and Mapping survey (OCCAM -\citealt{Frinchaboy2003},\citealt{Donor2018}). The recovery fraction is the fraction of open clusters that were matched to a group found by DBSCAN with high homogeneity and completeness. Currently, there are not enough open cluster members to make this recovery fraction a viable statistic, but in future the recovery fraction will provide a way to measure success of clustering on real data. In particular, I plan to revisit to the recovery fraction once open cluster data is publicly released from the GALAH spectroscopic survey \citep{DeSilva2015}.
	
	Having developed my metrics for DBSCAN success and drafted a paper describing applications to synthetic data, I have begun looking forward to future work. In particular, I have considered how best to apply DBSCAN to observations; this and some of the associated potential pitfalls are described in detail in the next section.
    
    In addition to the rest of my work, I attended the Gaia Sprint in NYC, where I developed the beginning of the framework I will use to apply my clustering algorithm to real data. In particular, I focussed testing alternate methods of filling in gaps where observations were missing or untrustworthy. I presented my work on clustering finding in simulated chemical spaces at the Dark Energy Survey workshop on near-field cosmology at the Kavli Institute for Cosmological Physics. I also applied and was accepted to the Life and Times of the Milky Way conference hosted by the Shanghai Astronomical Observatory, where I will be giving a contributed talk on my upcoming paper.
  
    
    \subsection*{Upcoming work and anticipated challenges}
    
    My main focus before the next meeting will be submitting and responding to comments on my paper describing the success of my cluster-finding algorithm on simulated data. I will also be working on ways to make my simulation even more realistic, primarily by incorporating a better model for the Milky Way than the uniform disk. I plan to include this on a future paper that compares the results of these more detailed simulations with the product of DBSCAN on observations from the APOGEE and GALAH surveys.
    
    While I've already developed approaches to handling the missing parts of observations so DBSCAN can still be used, I anticipate some teething problems when first applying the algorithm to the real data. An additional challenge will be extending DBSCAN to larger data sets, as its currently topping out at about 50,000 stars. While we may initially split observed samples into kinematically-identified components of the Galaxy like the halo and disk, it may eventually be useful to be able to interpret DBSCAN results on larger data sets. It seems that go further I will need to explore the backend of the code, or write my own version that has less stringent memory requirements.
	
   
 
\section*{DBSCAN applied to real data: paper outline}

One of my main goals for this thesis is to apply DBSCAN to real data, making use of the abundances and the spectra as separate chemical spaces. Having already prepared a paper on expectations for DBSCAN results from simulated data, I will be free in this work to demonstrate what we can learn from the observations. The paper will primarily focus on presenting the largest groups found by DBSCAN in the observations, with special attention paid to known open and globular clusters as checks for successful chemical space recovery. I plan to investigate a range of DBSCAN parameters to find the best ones for cluster recovery according to the measures of success laid out in my paper on applications to synthetic data. Although there may not be enough open cluster data to measure the fraction of known clusters recovered, I can seed my simulated clusters into the real data to serve as the known clusters. I will also check internally for the robustness of the groups found by DBSCAN by investigating whether membership is consistent across a range of DBSCAN parameters and whether stars grouped with the same set of other stars in different chemical spaces (e.g. abundances vs spectra0> Once a robust set of clusters is found from the real data, I will investigate the spatial and Hertzsprung-Russell locations of the clustered stars with help from 2MASS \citep{Skrutskie2006} and Gaia \citep{GaiaCollaboration2016}. If an individual cluster is promising, I may also attempt isochrone fitting to determine cluster age \citep{Bressan2012}. In addition to potentially using simulated clusters to measure recovery fraction, I can also use the simulations to make qualitative and quantitative comparisons to the found clusters. For this work I plan to upgrade my simulations, improving the Galaxy's density profile and the treatment of radial migration when considering cluster sampling rate.


\section*{Timeline}

This version of the timeline is modified from the original appearing in the thesis proposal. Temporally nearby goals are delineated in detail, while more distant goals are summarized as a series of papers. 

\subsection*{Detailed upcoming timeline}
Colorcoding is as follows:
    
    \begin{itemize}
    	\item {\color{RoyalBlue}blue} = completed goal
    	\item {\color{ForestGreen} green} = in progress goal
    	\item {\color{BurntOrange} orange} = replaced goal
    \end{itemize}

Goals that were completed or replaced at the last committee report have been removed

\begin{itemize}

\item By May 2018
\begin{itemize}
\item {\color{ForestGreen} write up a paper describing the cluster-finding algorithm and indicating which of label and spectral space is better for recovering clusters with high efficiency}
\item {\color{RoyalBlue}using APOGEE parameter distributions and observed open clusters, develop observational constraints on birth clusters parameters and the background distribution of stars that are the sole members of their birth clusters}
\begin{itemize}
\item {\color{RoyalBlue} use open cluster observations to set upper limits of birth cluster chemical homogeneity and uniqueness}
\item use theoretical models of stellar migration to predict how many stars are likely to be observed from a single birth cluster given an overall sample size and survey radius (e.g. \citealt{Ting2015})
\item {\color{RoyalBlue} use a realistic cluster mass function to create a chemical space with a natural background due to single member clusters}
\end{itemize}
\item \emph{New goal}: {\color{RoyalBlue} consider how cluster finding will work for different chemical space dimensionalities}
\item \emph{New goal}: {\color{RoyalBlue} make a decision about how to use DBSCAN on missing data}
\end{itemize}
\item By September 2018
\begin{itemize}
\item {\color{RoyalBlue}apply the cluster-finding algorithm to a realistic synthetic chemical space} 
\item {\color{RoyalBlue} explore the limits on the cluster creation parameters that will allow synthetic clusters to be recovered with reasonably high efficiency}
\item {\color{RoyalBlue} test clustering finding on observed data set with known open clusters}
\end{itemize}
\item By December 2018
\begin{itemize}
\item {\color{BurntOrange} write up a paper describing the parameter limits derived for synthetic clusters, and compare these with the limits derived by looking at open clusters} - this paper now incorporated into initial cluster-finding paper
\item investigate the likelihood that these limits hold for real star formation with this open cluster comparison, and tie this into theories of Milky Way evolution (rate of cluster disruption, GMC enrichment and stellar migration)
\begin{itemize}
\item \emph{New goal}: compare with APOGEE OCCAM sample
\end{itemize}
\item \emph{New goal}: apply DBSCAN to APOGEE and GALAH data
\begin{itemize}
\item set up simulation to create GALAH like results
\end{itemize}
\item \emph{New goal}: more explicitly include radial migration and a Milky Way density profile in the simulated clusters
\item \emph{New goal}: extend DBSCAN to larger samples
\end{itemize}
\item By May 2019
\begin{itemize}
\item use derived parameter limits to determine which surveys are likely to succeed in the limit of strong chemical tagging
	\begin{itemize}
		\item {\color{ForestGreen} generate results for a variety of sampling rates}
	\end{itemize}
\item {\color{RoyalBlue} check that real open clusters are correctly identified by the algorithm when examined in spectral and label space with other observed stars serving as a potential source of background}
\item {\color{ForestGreen} apply the cluster finding algorithm to observations from the APOGEE survey, beginning by using the principal component chemical space derived in \citet{Price-Jones2018}}
\item {\color{ForestGreen} compare group membership when using principal components derived from the red clump vs red giant samples}
\item \emph{New goal}: write up paper on applications of DBSCAN to real data
\end{itemize} 
\end{itemize}

\subsection*{Planned papers and associated goals}
\begin{itemize}
	\item Comparison of DBSCAN results on real data to simulations including the effects of radial migration, a realistic Milky Way model and a variety of survey selection functions
	\begin{itemize}
	\item choose and simulate an appropriate Milky Way model, and choose different selection functions to sample it
	\item incorporate radial migration into sampling rate when creating synthetic clusters
	\item apply DBSCAN to real data
	\end{itemize}
	\item Constrain the Milky Way CMF using clustering algorithms on APOGEE data
	\begin{itemize}
	\item include validation of clustering on observed open clusters
	\item simulate chemical spaces with a variety of CMFs and apply cluster finding to find the which space yields clusters most similar to those found in APOGEE data
	\end{itemize}
	\item Predict chemical tagging capabilities of upcoming spectroscopic surveys like Sloan 5 (\url{http://www.sdss.org/future/})
	\begin{itemize}
	\item predict the dimensionality of the chemical space such surveys sample and the resolution of that sampling
	\item extend the simulation of synthetic clusters to account for a greater number of stars and larger survey volume
	\end{itemize}
\end{itemize}
\bibliographystyle{apj}
\bibliography{sim}


\end{document}