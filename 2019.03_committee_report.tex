
% Default to the notebook output style

    


% Inherit from the specified cell style.




    
\documentclass[11pt]{article}

    \usepackage{natbib}
    \usepackage[usenames, dvipsnames]{color}
    %\bibliographystyle{apj}
    
    \usepackage[T1]{fontenc}
    % Nicer default font (+ math font) than Computer Modern for most use cases
    \usepackage{mathpazo}

    % Basic figure setup, for now with no caption control since it's done
    % automatically by Pandoc (which extracts ![](path) syntax from Markdown).
    \usepackage{graphicx}

    \usepackage{geometry} % Used to adjust the document margins
    \usepackage{amsmath} % Equations
    \usepackage{amssymb} % Equations
    
    \usepackage[mathletters]{ucs} % Extended unicode (utf-8) support
    \usepackage[utf8x]{inputenc} % Allow utf-8 characters in the tex document
    \usepackage{fancyvrb} % verbatim replacement that allows latex
    \usepackage{grffile} % extends the file name processing of package graphics 
                         % to support a larger range 
    % The hyperref package gives us a pdf with properly built
    % internal navigation ('pdf bookmarks' for the table of contents,
    % internal cross-reference links, web links for URLs, etc.)
    \usepackage{hyperref}
    \usepackage[normalem]{ulem} % ulem is needed to support strikethroughs (\sout)
                                % normalem makes italics be italics, not underlines
    
    \usepackage{fancyhdr}
    \usepackage{subcaption}  
    \usepackage{wrapfig,booktabs}
 
	\setcounter{topnumber}{2}
	\setcounter{bottomnumber}{2}
	\setcounter{totalnumber}{4}
	\renewcommand{\topfraction}{0.85}
	\renewcommand{\bottomfraction}{0.85}
	\renewcommand{\textfraction}{0.15}
	\renewcommand{\floatpagefraction}{0.8}
	\renewcommand{\textfraction}{0.1}
	\setlength{\floatsep}{5pt plus 2pt minus 2pt}
	\setlength{\textfloatsep}{5pt plus 2pt minus 2pt}
	\setlength{\intextsep}{2.5pt plus 0pt minus 0pt}
    
    \geometry{verbose,tmargin=1in,bmargin=1in,lmargin=1in,rmargin=1in}
    
    
    \pagestyle{fancy}
    \fancyhf{}
    \rhead{Natalie Price-Jones}
    \lhead{March 2019 Committee Report}
    \rfoot{Page \thepage}


    \begin{document}
    
    \section*{Committee Report}
    
    The primary goal of my PhD work is to answer questions about the Milky Way's formation and evolution using chemical tagging. Since my last meeting, I have completed and submitted a paper demonstrating that it is possible to recover realistic simulated clusters with a density-based approach to clustering in chemical space.    

    \subsection*{Degree progress}

	My primary focus during the past six months has been finalizing and submitting a paper describing the results of applying DBSCAN (density based spatial clustering applications with noise - \citealt{Ester1996}) to simulated chemical data. This data emulates results from H-band spectroscopic survey APOGEE (Apache Point Observatory Galactic Evolution Experiment - \citealt{Majewski2017}), but in the paper we make predictions for other chemical surveys, like GALAH \citep{DeSilva2015} and the Milky Way Mapper, a successor to APOGEE in SDSS-V \citep{Kollmeier2017}. We do this by creating a sample of stars and assigning them to birth clusters, giving stars from the same birth cluster the same chemistry. The size of each birth cluster was chosen according to a cluster mass function in accordance with observations.
	
	We tested the ability of DBSCAN  to recover these birth clusters using different kinds of chemical spaces, making a distinction between chemical spaces based on stellar spectra and those based on derived chemical abundances. In the case of the spectra, we showed that we can only do useful classification if we take steps to remove pixels that do not contribute chemical information (e.g. noise or continuum). We did this using a principal component analysis (PCA) reduction on the spectra \citep{Joliffe2002}. We explored abundance spaces with a few different choices for observational uncertainties: those typical for APOGEE's pipeline \citep{Abolfathi2018}, those typical for the \texttt{astroNN} neural network \citep{Leung2019}, and those predicted as theoretical best recovery in \citep{Ting2016b}. When using the latter two choices for abundance uncertainties or the PCA-reduced spectra we showed that we recovered between 20\% and 60\% of our simulated birth clusters with a homogeneity of at least 0.7 (cluster members made up 70\% of the group found by DBSCAN) and a completeness of at least 0.7 (70\% of cluster members were in the group found by DBSCAN). 
	
	As a preliminary test of DBSCAN on real cluster data, we included real APOGEE measurements for members of open clusters as identified by the Open Cluster Chemical Abundances and Mapping survey (OCCAM -\citealt{Frinchaboy2003}, \citealt{Donor2018}). Our strict quality cuts ensured each open cluster member had good measurements for each abundance of interest, but reduced the number of members enough that we were not able to recover any known open clusters. We were however able to assess how the open clusters compared to our simulated birth clusters using a metric I will apply to clusters recovered in real data; the silhouette coefficient. This metric increases as a cluster becomes more separated from surrounding clusters in chemical space. When examining our principal component chemical space, we found that the open clusters had a higher silhouette coefficient than the birth clusters we had created. This might be due to the low membership of the open clusters causing them to appear more separated than they would truly be with additional members; however they need only be at least as separated as our simulated birth clusters to be recovered when their membership is high enough.
	
	My next steps will be to investigate combinations of elements for which our cuts on open clusters need not be so stringent, and to apply DBSCAN to a larger set of observations. I have taken steps towards this by using DBSCAN on abundances for APOGEE stars derived by \texttt{astroNN} \citep{Leung2019}, where quality cuts leave a sample of about 40,000 stars. While clusters can be recovered, there are fewer than expected given the sample size, and I do not successfully recover any of the open clusters in the data set. However I continue to investigate what I have recovered, as well as how I may be able to work with even larger sets of observations.
	
	I was able to present some of the results described above at the Life and Times of the Milky Way conference in Shanghai in November.
    
    \subsection*{Upcoming work and anticipated challenges}
    
    Having only recently received the referee comments on my paper, my immediate future work will be updating my draft and submitting a response to the comments. However, now that I have taken the first steps towards applying my approach to real data, I am keen to continue exploring how DBSCAN recovered groups relate to birth clusters of stars. 
    
    As mentioned in previous reports, applying DBSCAN to observations will mean making choices about how to preprocess those observations, since a using Euclidean distance metric requires that each star have full chemical space information. I have mentioned in my main report that one way around this might be reducing the number of abundances considered; another would be to choose alternative ways to compute distance. For this reason I have begun working with Itamar Reis (Tel Aviv University) on doing classifications using Random Forest similarities \citep{Reis2018} instead of a Euclidean distance to perform clustering.
    
    I am also presently developing a way to partition larger datasets into sensible sections so they can be classified with DBSCAN. I am currently finding it challenging to prove that the same clusters are found in both the partitioned and unpartitioned data, which will be essential if I wish to justify sectioning off a large data set.
 
\section*{DBSCAN applied to real data: paper outline}

I will present the results of applying DBSCAN to observations in my next paper, with a particular focus on the largest clusters I recover. My work with my simulated birth clusters has indicated a good starting place for choosing an appropriate parameterization of DBSCAN. If I am unable to recover real open clusters, I will seed in simulated clusters with different silhouette coefficients to determine how large and well separated a cluster would need to be in order to be recovered by DBSCAN. I will confirm that clusters I find retain the same membership over small variations in the parameterization of DBSCAN, and check whether the clusters are the same when using DBSCAN on different chemical spaces. Once I have found a set of clusters with robust membership, I will explore their kinematic properties to understand how populations of stars with the same chemistry evolve throughout the Milky Way. The work done in this paper will inform how I update my simulations in future to make more detailed predictions for upcoming surveys.


\section*{Timeline}

This version of the timeline is modified from the original appearing in the thesis proposal. 

\subsection*{Detailed upcoming timeline}
Colorcoding is as follows:
    
    \begin{itemize}
    	\item {\color{RoyalBlue}blue} = completed goal
    	\item {\color{ForestGreen} green} = in progress goal
    	\item {\color{BurntOrange} orange} = replaced goal
    \end{itemize}

Goals that were completed or replaced at the last committee report have been removed

\begin{itemize}

\item By December 2018
\begin{itemize}
\item {\color{RoyalBlue} write up a paper describing the parameter limits derived for synthetic clusters, and compare these with the limits derived by looking at open clusters}
\item {\color{ForestGreen} investigate the likelihood that these limits hold for real star formation with this open cluster comparison, and tie this into theories of Milky Way evolution (rate of cluster disruption, GMC enrichment and stellar migration)}
\begin{itemize}
\item {\color{RoyalBlue} compare with APOGEE OCCAM sample}
\end{itemize}
\item {\color{ForestGreen} apply DBSCAN to APOGEE and GALAH data}
\begin{itemize}
\item set up simulation to create GALAH like results
\end{itemize}
\item {\color{RoyalBlue} more explicitly include radial migration and a Milky Way density profile in the simulated clusters}
\item {\color{ForestGreen} extend DBSCAN to larger samples}
\end{itemize}
\item By May 2019
\begin{itemize}
\item {\color{RoyalBlue} use derived parameter limits to determine which surveys are likely to succeed in the limit of strong chemical tagging}
	\begin{itemize}
		\item {\color{RoyalBlue} generate results for a variety of sampling rates}
	\end{itemize}
\item {\color{RoyalBlue} check that real open clusters are correctly identified by the algorithm when examined in spectral and label space with other observed stars serving as a potential source of background} 
\item {\color{RoyalBlue} apply the cluster finding algorithm to observations from the APOGEE survey}
\end{itemize}

\item By September 2019
\begin{itemize}
\item {start writing up a paper describing the results of cluster finding on observed clusters and on principal component chemical space} - changed to now focus on abundances as well as PCA space.
\item {\color{ForestGreen} apply my algorithm to the APOGEE red clump sample, using positional information to investigate how chemically similar populations of stars are distributed throughout the Galaxy} - changed; now starting with full APOGEE sample, using Gaia for kinematic information.
\item \emph{New goal:} relate my findings when applying to APOGEE and GALAH to theories of Galactic chemical and dynamical evolution.
\end{itemize}
\item By December 2019
\begin{itemize}
\item \emph{New goal:} write introductory thesis chapter explaining how my three papers (chemical space dimensionality, DBSCAN results on simulated data, and DBSCAN results on real data) all fit together
\item \emph{New goal:} likely continue work on paper describing the application of DBSCAN to real data
\item \emph{New goal:} find external committee member for thesis defense
\item \emph{New goal:} apply more focus to making connections for a potential industry position
\end{itemize}
\end{itemize}

What follows is a detailed timeline to track deadlines for completing my degree on time.
\begin{itemize}
\item By April 2020
\begin{itemize}
	\item choose external examiner
	\item begin to construct final thesis
\end{itemize}
\item By June 2020
\begin{itemize}
	\item submit thesis to external examiner (no later than June 1)
	\item submit thesis to final oral exam committee and graduate administrator (no later than June 15 and ideally by June 1)
\end{itemize}
\item By July 2020
\begin{itemize}
\item submit final thesis title to graduate administrator (no later than July 1)
\item ensure external examiner's thesis appraisal report is collected (no later than July 15)
\end{itemize}
\item During August 2020
\begin{itemize}
\item defend thesis
\end{itemize}
\end{itemize}

\subsection*{Planned papers and associated goals}
\begin{itemize}
	\item Presentation of predictions for future surveys with chemical tagging as a goal based on improved simulations including the effects of radial migration, a realistic Milky Way model and a variety of survey selection functions
	\begin{itemize}
	\item research relevant selection functions/survey techniques for the Milky Way Mapper in SDSS-V \citep{Kollmeier2017} and the Mauna Kea Spectroscopic Explorer \citep{Zhang2018}
	\item choose and simulate an appropriate Milky Way model
	\item directly include the effects of radial migration when choosing which stars are sampled from original clusters
	\end{itemize}
	\item Constrain the Milky Way CMF using clustering algorithms on APOGEE data
	\begin{itemize}
	\item include validation of clustering on observed open clusters
	\item simulate chemical spaces with a variety of CMFs and apply cluster finding to find the which space yields clusters most similar to those found in APOGEE data
	\end{itemize}
\end{itemize}
\bibliographystyle{apj}
\bibliography{sim}


\end{document}