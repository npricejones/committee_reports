
% Default to the notebook output style

    


% Inherit from the specified cell style.




    
\documentclass[11pt]{article}

    \usepackage{natbib}
    \usepackage[usenames, dvipsnames]{color}
    %\bibliographystyle{apj}
    
    \usepackage[T1]{fontenc}
    % Nicer default font (+ math font) than Computer Modern for most use cases
    \usepackage{mathpazo}

    % Basic figure setup, for now with no caption control since it's done
    % automatically by Pandoc (which extracts ![](path) syntax from Markdown).
    \usepackage{graphicx}

    \usepackage{geometry} % Used to adjust the document margins
    \usepackage{amsmath} % Equations
    \usepackage{amssymb} % Equations
    
    \usepackage[mathletters]{ucs} % Extended unicode (utf-8) support
    \usepackage[utf8x]{inputenc} % Allow utf-8 characters in the tex document
    \usepackage{fancyvrb} % verbatim replacement that allows latex
    \usepackage{grffile} % extends the file name processing of package graphics 
                         % to support a larger range 
    % The hyperref package gives us a pdf with properly built
    % internal navigation ('pdf bookmarks' for the table of contents,
    % internal cross-reference links, web links for URLs, etc.)
    \usepackage{hyperref}
    \usepackage[normalem]{ulem} % ulem is needed to support strikethroughs (\sout)
                                % normalem makes italics be italics, not underlines
    
    \usepackage{fancyhdr}
    \usepackage{subcaption}  
    \usepackage{wrapfig,booktabs}
 
	\setcounter{topnumber}{2}
	\setcounter{bottomnumber}{2}
	\setcounter{totalnumber}{4}
	\renewcommand{\topfraction}{0.85}
	\renewcommand{\bottomfraction}{0.85}
	\renewcommand{\textfraction}{0.15}
	\renewcommand{\floatpagefraction}{0.8}
	\renewcommand{\textfraction}{0.1}
	\setlength{\floatsep}{5pt plus 2pt minus 2pt}
	\setlength{\textfloatsep}{5pt plus 2pt minus 2pt}
	\setlength{\intextsep}{2.5pt plus 0pt minus 0pt}
    
    \geometry{verbose,tmargin=1in,bmargin=1in,lmargin=1in,rmargin=1in}
     
    \pagestyle{fancy}
    \fancyhf{}
    \rhead{Natalie Price-Jones}
    \lhead{March 2020 Committee Report}
    \rfoot{Page \thepage}


    \begin{document}
    
    \section*{Committee Report}


    \subsection*{Degree progress}

    
    \subsection*{Upcoming work and anticipated challenges}
    


\section*{Timeline}

This version of the timeline is modified from the original appearing in the thesis proposal. 

\subsection*{Detailed upcoming timeline}
Color-coding is as follows:
    
    \begin{itemize}
    	\item {\color{RoyalBlue}blue} = completed goal
    	\item {\color{ForestGreen} green} = in progress goal
    \end{itemize}

Goals that were completed or replaced at the last committee report have been removed

\begin{itemize}

\item By December 2019
\begin{itemize}
\item {\color{ForestGreen} write introductory thesis chapter explaining how my three papers (chemical space dimensionality, DBSCAN results on simulated data, and DBSCAN results on real data) all fit together}
\item {\color{RoyalBlue}finish paper describing application of DBSCAN to APOGEE data}
\item {\color{RoyalBlue} find external committee member for thesis defense}
\item {\color{ForestGreen} apply more focus to making connections for a potential industry position}
\end{itemize}
\item May 2020
\begin{itemize}
\item {complete any revisions to paper describing applications of DBSCAN to APOGE data}
\item {\color{ForestGreen} combine introduction with my publications to form a thesis}
\item {submit thesis to external examiner (no later than May 10)}
\item submit thesis to final oral exam committee and graduate administrator (no later than May 24 and ideally by May 1)
\end{itemize}
\item June 2020
\begin{itemize}
\item submit final thesis title to graduate administrator (no later than June 10)
\item ensure external examiner's thesis appraisal report is collected (no later than June 24) 
\end{itemize}
\item July 10 2020 - Defend Thesis
\end{itemize}

\bibliographystyle{apj}
\bibliography{sim}


\end{document}
